%% start of file `template.tex'.
%% Copyright 2006-2015 Xavier Danaux (xdanaux@gmail.com), 2020-2022 moderncv maintainers (github.com/moderncv).
%
% This work may be distributed and/or modified under the
% conditions of the LaTeX Project Public License version 1.3c,
% available at http://www.latex-project.org/lppl/.

\documentclass[11pt,a4paper,sans]{moderncv}        % possible options include font size ('10pt', '11pt' and '12pt'), paper size ('a4paper', 'letterpaper', 'a5paper', 'legalpaper', 'executivepaper' and 'landscape') and font family ('sans' and 'roman')

% moderncv themes
\moderncvstyle{casual}                             % style options are 'casual' (default), 'classic', 'banking', 'oldstyle' and 'fancy'
\moderncvcolor{blue}                               % color options 'black', 'blue' (default), 'burgundy', 'green', 'grey', 'orange', 'purple' and 'red'
%\renewcommand{\familydefault}{\sfdefault}         % to set the default font; use '\sfdefault' for the default sans serif font, '\rmdefault' for the default roman one, or any tex font name
%\nopagenumbers{}                                  % uncomment to suppress automatic page numbering for CVs longer than one page

% adjust the page margins
\usepackage[scale=0.85]{geometry}
\setlength{\footskip}{6.00005pt}                 % depending on the amount of information in the footer, you need to change this value. comment this line out and set it to the size given in the warning
%\setlength{\hintscolumnwidth}{3cm}                % if you want to change the width of the column with the dates
%\setlength{\makecvheadnamewidth}{10cm}            % for the 'classic' style, if you want to force the width allocated to your name and avoid line breaks. be careful though, the length is normally calculated to avoid any overlap with your personal info; use this at your own typographical risks...

% font loading
% for luatex and xetex, do not use inputenc and fontenc
% see https://tex.stackexchange.com/a/496643
\ifxetexorluatex
  \usepackage{fontspec}
  \usepackage{unicode-math}
  \defaultfontfeatures{Ligatures=TeX}
  \setmainfont{Latin Modern Roman}
  \setsansfont{Latin Modern Sans}
  \setmonofont{Latin Modern Mono}
  \setmathfont{Latin Modern Math} 
\else
  \usepackage[utf8]{inputenc}
  \usepackage[T1]{fontenc}
  \usepackage{lmodern}
\fi

% document language
\usepackage[english]{babel}  % FIXME: using spanish breaks moderncv
\usepackage{csquotes}
\usepackage{amsmath}

% personal data
\name{Marwan}{Benyoussef}
\title{PhD student at Freie Universit\"{a}t Berlin}                               % optional, remove / comment the line if not wanted
\born{19 Sept 1983}                                 % optional, remove / comment the line if not wanted
\address{Wasgenstr. 75, Haus 25}{14129, Berlin}{Germany}% optional, remove / comment the line if not wanted; the "postcode city" and "country" arguments can be omitted or provided empty
\phone[mobile]{+33 767813990}                   % optional, remove / comment the line if not wanted; the optional "type" of the phone can be "mobile" (default), "fixed" or "fax"
%\phone[fixed]{+2~(345)~678~901}
%\phone[fax]{+3~(456)~789~012}
\email{marwan.benyoussef@gmail.com}                               % optional, remove / comment the line if not wanted
\homepage{https://marwanus.github.io/}                         % optional, remove / comment the line if not wanted

% Social icons
%\social[linkedin]{john.doe}                        % optional, remove / comment the line if not wanted
%\social[xing]{john\_doe}                           % optional, remove / comment the line if not wanted
%\social[twitter]{ji\_doe}                             % optional, remove / comment the line if not wanted
%\social[github]{jdoe}                              % optional, remove / comment the line if not wanted
%\social[gitlab]{jdoe}                              % optional, remove / comment the line if not wanted
%\social[stackoverflow]{0000000/johndoe}            % optional, remove / comment the line if not wanted
%\social[bitbucket]{jdoe}                           % optional, remove / comment the line if not wanted
%\social[skype]{jdoe}                               % optional, remove / comment the line if not wanted
%\social[orcid]{0000-0000-000-000}                  % optional, remove / comment the line if not wanted
%\social[researchgate]{jdoe}                        % optional, remove / comment the line if not wanted
%\social[researcherid]{jdoe}                        % optional, remove / comment the line if not wanted
%\social[telegram]{jdoe}                            % optional, remove / comment the line if not wanted
%\social[whatsapp]{12345678901}                     % optional, remove / comment the line if not wanted
%\social[signal]{12345678901}                       % optional, remove / comment the line if not wanted
%\social[matrix]{@johndoe:matrix.org}               % optional, remove / comment the line if not wanted
%\social[googlescholar]{googlescholarid}            % optional, remove / comment the line if not wanted


\extrainfo{additional information}                 % optional, remove / comment the line if not wanted
%\photo[64pt][0.4pt]{picture}                       % optional, remove / comment the line if not wanted; '64pt' is the height the picture must be resized to, 0.4pt is the thickness of the frame around it (put it to 0pt for no frame) and 'picture' is the name of the picture file
%\quote{Some quote}                                 % optional, remove / comment the line if not wanted

% bibliography adjustments (only useful if you make citations in your resume, or print a list of publications using BibTeX)
%   to show numerical labels in the bibliography (default is to show no labels)
%\makeatletter\renewcommand*{\bibliographyitemlabel}{\@biblabel{\arabic{enumiv}}}\makeatother
\renewcommand*{\bibliographyitemlabel}{[\arabic{enumiv}]}
%   to redefine the bibliography heading string ("Publications")
%\renewcommand{\refname}{Articles}

% bibliography with mutiple entries
%\usepackage{multibib}
%\newcites{book,misc}{{Books},{Others}}
%----------------------------------------------------------------------------------
%            content
%----------------------------------------------------------------------------------
\begin{document}
%\begin{CJK*}{UTF8}{gbsn}                          % to typeset your resume in Chinese using CJK
%-----       resume       ---------------------------------------------------------
\makecvtitle
\vspace*{-1cm}

%\hypersetup{
%	pdfborderstyle={/S/U/W 1}, % underline links instead of boxes
%	linkbordercolor=red,       % color of internal links
%	citebordercolor=green,     % color of links to bibliography
%	filebordercolor=magenta,   % color of file links
%	urlbordercolor=blue        % color of external links
%}

\section{Education}
\cventry{2018-2020}{Master of Science in Mathematics, with honours}{Sorbonne University}{}{\textit{{}{\textit{Paris (with one semester exchange at University of Montreal)}}{}}}{}
\cventry{2018}{Bachelor of Science in Mathematics, with honors}{Sorbonne University}{Paris}{\textit{{}{\textit{France}}{}}}{}
\cventry{2007}{Embedded Systems Engineer}{Ecole des Mines de Saint-Etienne}{Saint-Etienne}{\textit{France}}{}  % arguments 3 to 6 can be left empty


\section{Languages}
\cvitemwithcomment{French}{Native.}{}
\cvitemwithcomment{Arabic}{Native.}{}
\cvitemwithcomment{English}{Fluent.}{}
\cvitemwithcomment{German}{Beginner, obtained an $ A_2 $-level certificate.}{}

\section{Computer skills}
%\cvdoubleitem{Algebra:}{Commutative Algebra, Galois Theory, Group Representations, Homological Algebra}{Geometry:}{Algebraic Geometry, Differential Geometry, Lie Groups and Algebras, De Rham cohomology, Čech cohomology, Descent Theory, D-modules, Triangulated and Derived Categories, Resolution of singularities in characteristic 0.}
%\cvdoubleitem{Number Theory:}{Modular Arithmetic, Elliptic curves, Modular Forms}{Analysis:}{Sobolev Spaces, spectral theory of compact operators}
\cvdoubleitem{Operating Systems:}{Windows, Unix, Linux, HP, Aix, TPF-IBM, Vmware, virtualBox}{Tools:}{Scipy, mathplotlib, scikit-learn, Eclipse, Xemacs, LabView, Valgrind, Gdb}
\cvdoubleitem{Config Management:}{CVS, Mercurial, Git}{Design:}{Rational Rose, StarUML, Design Patterns, ISIS, Ares}
\cvdoubleitem{Languages:}{SageMath, GAP,  C/Embedded C, C++, Qt, SQL, Latex, PHP, Shell, Perl, Python, Ruby, AngularJS, UML}{\textit{Continuous} : \textit{Integration}}{GitLab, Jenkins, hudson, Maven, Ant}


%\section{Skill matrix}
%\cvitem{Skill matrix}{Alternatively, provide a skill matrix to show off your skills}
%%% Skill matrix as an alternative to rate one's skills, computer or other. 
%
%%% Adjusts width of skill matrix columns. 
%%% Usage \setcvskillcolumns[<width>][<factor>][<exp_width>]
%%% <width>, <exp_width> should be lengths smaller than \textwidth, <factor> needs to be between 0 and 1.
%%% Examples:
%% \setcvskillcolumns[5em][][]%    adjust first column. Same as \setcvskillcolumns[5em]
%% \setcvskillcolumns[][0.45][]%   adjust third (skill) column. Same as \setcvskillcolumns[][0.45]
%% \setcvskillcolumns[][][\widthof{``Year''}]%     adjust fourth (years) column.
%% \setcvskillcolumns[][0.45][\widthof{``Year''}]%
%% \setcvskillcolumns[\widthof{``Languag''}][0.48][]
%% \setcvskillcolumns[\widthof{``Languag''}]%
%
%%% Adjusts width of legend columns. Usage \setcvskilllegendcolumns[<width>][<factor>]
%%% <factor> needs to be between 0 and 1. <width> should be a length smaller than \textwidth
%%% Examples:
%% \setcvskilllegendcolumns[][0.45]
%% \setcvskilllegendcolumns[\widthof{``Legend''}][0.45]
%% \setcvskilllegendcolumns[0ex][0.46]% this is usefull for the banking style
%
%%% Add a legend if you are using \cvskill{<1-5>} command or \cvskillentry
%%% Usage \cvskilllegend[*][<post_padding>][<first_level>][<second_level>][<third_level>][<fourth_level>][<fifth_level>]{<name>}
%% \cvskilllegend % insert default legend without lines
%\cvskilllegend*[1em]{}% adjust post spacing
%% \cvskilllegend*{Legend}%  Alternatively add a description string
%%% adjust the legend entries for other languages, here German
%% \cvskilllegend[0.2em][Grundkenntnisse][Grundkenntnisse und eigene Erfahrung in Projekten][Umfangreiche Erfahrung in Projekten][Vertiefte Expertenkenntnisse][Experte\,/\,Spezialist]{Legende}
%
%%% Alternative legend style with the first three skill levels in one column
%%% Usage \cvskillplainlegend[*][<post_padding>][<first_level>][<second_level>][<third_level>][<fourth_level>][<fifth_level>]{<name>}
%% \setcvskilllegendcolumns[][0.6]%  works for classic, casual, banking
%% \setcvskilllegendcolumns[][0.55]%  works better for oldstyle and fancy
%% \cvskillplainlegend{}
%% \cvskillplainlegend[0.2em][Grundkenntnisse][Grundkenntnisse und eigene Erfahrung in Projekten][Umfangreiche Erfahrung in Projekten][Vertiefte Expertenkenntnisse][Experte/Guru]{Legende}
%
%%% Add a head of the skill matrix table with descriptions.
%%% Usage \cvskillhead[<post_padding>][<Level>][<Skill>][<Years>][<Comment>]%
%\cvskillhead[-0.1em]%   this inserts the standard legend in english and adjust padding
%%% Adjust head of the skill matrix for other languages
%% \cvskillhead[0.25em][Level][F\"ahigkeit][Jahre][Bemerkung]
%
%%% \cvskillentry[*][<post_padding>]{<skill_cathegory>}{<0-5>}{<skill_name>}{<years_of_experience>}{<comment>}% 
%%% Example usages:
%\cvskillentry*{Language:}{3}{Python}{2}{I'm so experienced in Python and have realised a million projects. At least.}
%\cvskillentry{}{2}{Lilypond}{14}{So much sheet music! Man, I'm the best!}
%\cvskillentry{}{3}{\LaTeX}{14}{Clearly I rock at \LaTeX}
%\cvskillentry*{OS:}{3}{Linux}{2}{I only use Archlinux btw}% notice the use of the starred command and the optional 
%\cvskillentry*[1em]{Methods}{4}{SCRUM}{8}{SCRUM master for 5 years}
%%% \cvskill{<0-5>} command
%% \cvitem{\textbackslash{cvskill}:}{Skills can be visually expressed by the \textbackslash{cvskill} command, e.g. \cvskill{2}}
%
%\section{Interests}
%\cvitem{hobby 1}{Description}
%\cvitem{hobby 2}{Description}
%\cvitem{hobby 3}{Description}
%
%\section{Extra 1}
%\cvlistitem{Item 1}
%\cvlistitem{Item 2}
%\cvlistitem{Item 3. This item is particularly long and therefore normally spans over several lines. Did you notice the indentation when the line wraps?}
%
%\section{Extra 2}
%\cvlistdoubleitem{Item 1}{Item 4}
%\cvlistdoubleitem{Item 2}{Item 5\cite{book2}}
%\cvlistdoubleitem{Item 3}{Item 6. Like item 3 in the single column list before, this item is particularly long to wrap over several lines.}

%\section{References}
%\begin{cvcolumns}
%  \cvcolumn{Category 1}{\begin{itemize}\item Person 1\item Person 2\item Person 3\end{itemize}}
%  \cvcolumn{Category 2}{Amongst others:\begin{itemize}\item Person 1, and\item Person 2\end{itemize}(more upon request)}
%  \cvcolumn[0.5]{All the rest \& some more}{\textit{That} person, and \textbf{those} also (all available upon request).}
%\end{cvcolumns}

\section{Master thesis}
\cvitem{title}{\emph{On the Hitchin morphism in positive characteristic}}
\cvitem{supervisor}{\href{https://imag.umontpellier.fr/~dos-santos/}{João Pedro P. dos Santos}}
\cvitem{description}{My thesis work is an exposition of some results relating the Hitchin fibration in positive characteristic and the theory of connections on vector bundles.
	The nice functorial properties of the sheaf of principal parts and the relative
	Frobenius morphism on a smooth projective curve entail categorical and
	geometric properties of the moduli stack of vector bundles with integrable
	connections.
	In particular, the stack of vector bundles with integrable connections is
	relatively representable, algebraic and of finite type over the affine line.
	Moreover, one can construct an interpolating object between a Higgs field
	and a connection, and define a special analogue of p-curvature, for which the
	characteristic polynomial factors through an affine subspace, and such that the
	fiber over zero coincides with the Hitchin morphism.}

\section{Teaching Experience}
	My duties range from designing the class content, grading the assignments and exams,  choosing the right problems to illustrate the mathematical concepts, to 
giving background and motivating lectures to the some central topics in algebraic geometry like scheme theory and moduli spaces. Moreover, I am very motivated to guide my students through the process of choosing their preferred future research area. For computer science teaching duties, I am also responsible for teaching my students how to use the software (C, C++, Maple, LaTeX) and how to conceive efficient algorithms and best design paradigms.
\subsection{Teaching assistant in Mathematics}
\cventry{Oct 2023 -- Present}{Algebra I (Graduate Level)}{Complex Analysis Department}{FU Berlin }{}{
	Conducting central exercise classes for the Algebra I module, covering concepts such as affine algebraic varieties, rings, ideals and modules, noetherian rings, local rings and localization, primary decomposition, finite and integral extensions, dimension theory, and regular rings.
}

\cventry{Oct 2023 -- Present}{Algebra and Number theory(Undergraduate Level)}{Mathematics and Computer Science Department}{FU Berlin }{}{
	Lecturing on topics including divisibility in rings (e.g., ring of integers and polynomial rings), residual classes and congruences, modules and ideals, Euclidean, principal ideal and factorial rings, the square reciprocity law, prime number testing and cryptography, structure of abelian groups, theorem on symmetric functions, field extensions, Galois correspondence, constructions with compass and ruler, non-abelian groups (Lagrange's theorem, normal subgroup, solvability, Sylow groups).
}

\cventry{Oct 2022 -- Mar 2023}{Algebra and Number Theory Teaching Assistant (Undergraduate Level)}{Complex Analysis Department}{FU Berlin }{}{
	Conducted exercise classes covering topics such as divisibility in rings, residual classes, congruences, modules, ideals, Euclidean, principal ideal and factorial rings, square reciprocity law, prime number testing, cryptography, structure of abelian groups, theorem on symmetric functions, field extensions, Galois correspondence, constructions with compass and ruler, non-abelian groups (Lagrange’s theorem, normal subgroup, solvability, Sylow groups).
}

\cventry{Apr 2022 -- Sep 2022}{Geometry Exercise Class (Undergraduate Level)}{Complex Analysis Department}{FU Berlin }{}{
	Conducted exercise sessions covering conic sections, transformations in Euclidean and affine geometry, projective geometry, and hyperbolic geometry.
}

\cventry{Oct 2021 -- Mar 2022}{Algebra III Exercise Class (Graduate Level)}{Complex Analysis Department}{FU Berlin }{}{
	Conducted exercise sessions covering Čech cohomology, schemes, cohomology of affine schemes, projective spaces, line bundles on projective spaces, their cohomology, group schemes, and algebraic groups.
}

\cventry{Apr 2021 -- Sep 2021}{Algebra II Exercise Class (Graduate Level)}{Complex Analysis Department}{FU Berlin }{}{
	Conducted exercise sessions covering categories and functors, additive and abelian categories, cohomology, ringed spaces, and sheaf theory.
}

\cventry{Oct 2020 -- Mar 2021}{Algebra I Exercise Class (Graduate Level)}{Complex Analysis Department}{FU Berlin }{}{
	Conducted central exercise class covering affine algebraic varieties, rings, ideals and modules, noetherian rings, local rings and localization, primary decomposition, finite and integral extensions, dimension theory, and regular rings.
}

\cventry{Jun 2018 -- Jun 2020}{Math Tutor}{Paris, France}{}{}{
	Provided tutoring for high school students preparing for \enquote{Classes préparatoires pour les grandes écoles }, focusing on undergraduate linear algebra and calculus for highly competitive examination entrance to engineering schools.
}


\subsection{Teaching assistant in Computer Science, Cybersecurity}
\cventry{Mar 2023 -- Sep 2023}{Cybersecurity Exercise Class (Graduate Level)}{Computer Science Department}{FU Berlin }{}{Conducted exercise class for Cybersecurity module, including the following concepts: authentication mechanisms, security models (Access Control Matrix Model, the Take-Grant Protection Model, the Bell-LaPadula and related models, the Chinese Wall Model, the Lattice Model of Information Flow), capability based systems and hardware protection mechanism concepts (protection rings). The course included also implementation vulnerabilities such as race conditions, stack and heap overflows, integer overflows, and return oriented programming.
	Problems that arise when humans interface with security e.g., password habits and password entry mechanisms, human responses to security prompts, incentives and distractors for better security, or reverse Turing tests.
}

\cventry{Sep 2017 -- Mar 2018}{Part-time Lecturer (Undergraduate Level)}{Mathematics and Computer Science Department}{Paris-Sud University}{}{Taught C++ and Object-Oriented Development and conducted lectures on algorithms and optimization.
}\newpage

\section{(Co-)organized seminars}
\subsection{Undergraduate Seminars}
	\cventry{Oct 2022 -- Mar 2023}{Proseminar on \enquote{Proofs from THE BOOK}}{Complex Analysis Department}{FU Berlin}{}{
		Conducted a proseminar on efficient, concise, and beautiful proofs in linear algebra, analysis, number theory, and discrete mathematics, inspired by Paul Erdos \enquote{The Proofs from THE BOOK}.
		The goal of the seminar is to bring undergraduate students to the level where can choose among different mathematical subjects a topic to present. I have accompagnied them through the process of choosing the important parts to present and to customize a ninety minutes talk, taking into account time management for the content and answering questions from the audience. During this seminar, I had highly motivated students who gave very diverse and successful talks, including 
		\begin{itemize}
			\item Approximation of integrals with tiling recltangles,
			\item The classical Cantor’s diagonal argument about the uncountability of the reals using binary trees for counting the ratioals,
			\item Subexponentially growing sequence of integers providing infinitely many proofs for the existence of infinitely many prime numbers,
			\item Buffons needle problem and probability theory to approximate $ \pi $, proving its irrationality,
			\item Using non-archimedean valuations to prove Monsky's theorem, stating the  impossibility to dissect a square into an odd number of triangles of equal area.
		\end{itemize}
	}



\subsection{Graduate Seminars}
\cventry{Winter Semester 2022}{Seminar on \enquote{Derived categories and the Mukai transform}}{Complex Analysis Department}{FU Berlin }{}{
		Homological algebra deals with complexes in abelian categories and their cohomology. This leads, among other things, to the concept of quasi-isomorphism between complexes. It is also observed that homotopic homomorphisms between complexes induce the same homomorphism on the cohomology. The derived  abelian category is obtained by forming the homotopy category of complexes in which the morphisms consist of homotopy equivalence classes of homomorphisms of a complex $ K $, and demanding that quasi-isomorphisms become isomorphisms. The last step in particular is technically complex. The derived category of an abelian category is no longer an abelian category. Short exact sequences are replaced by distinguished triangles. This concept is formalized by the concept of a triangulated categories.
		The known functors from homological algebra can now be elegantly introduced as functors between derived categories. The derived categories are very interesting
		objects. This is especially true for the derived categories of algebraic varieties, which arise from the abelian categories of coherent sheaves
		varieties. The seminar covered the basic notions of the theory of triangulated categories and the construction of derived categories. Special attention was paid to the derived categories of algebraic varieties. During this seminar, we covered chapters $ 1 $ to $ 3 $ from the book by Huybrechts \cite{huy06}.  Highly motivated students gave interesting talks delivering the following content: 
		\begin{itemize}
			\item Triangulated categories, exact functors and Serre functors,
			\item Equivalences of derived categories, exceptional collections, (semi-)orthogonal decompositions,
			\item Localisation of categories, the derived category of an abelian category,
			\item Bounded derived categories of an abelian category via injective and projective resolutions,
			\item Derived functors,
			\item Spectral sequences,
			\item The Grothendieck spectral sequence,
			\item Derived categories in algebraic geometry,
			\item Derived functors in algebraic geometry.
	\end{itemize}}
\subsection{Research Seminars}

\cventry{Summer Semester 2023}{Seminar on \enquote{Groebner bases}}{Complex Analysis Department}{FU Berlin}{}{
	\begin{itemize}
		\item Talk 1: Groebner bases, commutative and non-commutative cases.(\href{https://marwanus.github.io/documents/L-Talk1-Grobner_bases.pdf}{notes})
		\item Talk 2: Hochschield cohomology.(\href{https://marwanus.github.io/documents/L-Talk2-Hochschield_cohomology.pdf}{notes})
		\item Talk 3: Exactness.(\href{https://marwanus.github.io/documents/L-Talk-3-Exactness.pdf}{notes})
	\end{itemize}
}

\cventry{Winter Semester 2023}{Seminar on \enquote{On the Hitchin morphism for higher dimensional varieties}}{Complex Analysis Department}{FU Berlin}{}{
The talks delivered the following topics.
	\begin{itemize}
		\item \href{https://marwanus.github.io/documents/Seminar\%20program_NGO_seminar.pdf}{Seminar program},
		\item Talk 1: Hitchin fibration for algebraic curves(\href{https://marwanus.github.io/documents/L-Talk1-Hitchin_fibration_algebraic_curves.pdf}{notes}).
		\item Talk 2: Spectral covers (\href{https://marwanus.github.io/documents/L-Talk2-Spectral-Covers.pdf}{notes})
		\item Talk 3: Cameral covers(\href{https://marwanus.github.io/documents/L-Talk3-Cameral_covers.pdf}{notes}).
		\item Talk 4: Representability Lemma (\href{https://marwanus.github.io/documents/L-Talk4-representability_lemma.pdf}{notes})
		\item Talk 5: Spectral data morphism and Hitchin map via Weyl polarization (\href{https://marwanus.github.io/documents/L-Talk5-spectral_data_morphism.pdf}{notes}),
		\item Talk 6: Cohen-Macaulay spectral surfaces (\href{https://marwanus.github.io/documents/L-Talk6-Cohen-Macaulay_spectral_surfaces.pdf}{notes}),
		\item Talk 7: Some examples and consequences (\href{https://marwanus.github.io/documents/L-Talk7-Some_examples.pdf}{notes}).
	\end{itemize}
}

\cventry{Summer Semester 2022}{Seminar on \enquote{Bridgeland Stability Conditions}}{Complex Analysis Department}{FU Berlin}{}{
	A semiar co-organized with my colleagues Juan Martin Perez Bernal, Cesare Goretti, Jan Marten Sevenster. The talks delivered the following topics.
	\begin{itemize}
		\item \href{https://marwanus.github.io/documents/program_Bridgeland_stability_SoSE22.pdf}{Seminar program}.
		\item Talk 1: The motivating example of $ \operatorname{Num}_X$.(\href{https://marwanus.github.io/documents/L-Talk1-The_motivating_example_of_Num_X.pdf}{notes})
		\item Talk 3: Bridgeland stability conditions.(\href{https://marwanus.github.io/documents/L-Talk3-Bridgeland_stability_condition.pdf}{notes})
		\item Talk 4: Moduli Spaces.(\href{https://marwanus.github.io/documents/L-Talk4-Moduli_spaces.pdf}{notes})
		\item Talk 5: Walls and chambers.(\href{https://marwanus.github.io/documents/L-Talk5-Walls_and_chambers.pdf}{notes})
	\end{itemize}
}


\cventry{Winter Semester 2022}{Seminar on \enquote{Infinite dimensional GIT}}{Complex Analysis Department}{FU Berlin}{}{
	A semiar co-organized with my colleague Juan Martin Perez Bernal, the talks delivered the following topics.
\begin{itemize}
	\item \href{https://marwanus.github.io/documents/Program_Infinite_dimensional_GIT.pdf}{Seminar program},
	\item Talk 1: Ind-schemes and affine grassmanians for $ \operatorname{GL}_n $(\href{https://marwanus.github.io/documents/L-Talk1-Affine\%20Grassmanians.pdf}{notes}).
			\item Talk 2: Corepresentability ofthe moduli functor of semistable bundles (\href{https://marwanus.github.io/documents/L-Talk2-Corepresentability_of_mod_functor.pdf}{notes})
			\item Talk 3: Theta instability theory (\href{https://marwanus.github.io/documents/L-Talk3-Theta-instability\%20theory.pdf}{notes}).
			\item Talk 5: Classical vs Infinite dimensional GIT (\href{https://marwanus.github.io/documents/L-Talk5-Inf_dim_GIT_vs_Classical.pdf}{notes})
			\item Talk 6: Rational filling for torsion-free sheaves (\href{https://marwanus.github.io/documents/L-Talk6-Rational\%20filling.pdf}{notes}),
			\item Talk 7: Theta stratifications  (\href{https://marwanus.github.io/documents/P-Talk7-theta_stratif_Lambda_coh.pdf}{notes}),
			\item Talk 8: The geometric template (\href{https://marwanus.github.io/documents/L-Talk8-Geometric\%20template.pdf}{notes}).
\end{itemize}
}


% Publications from a BibTeX file without multibib
%  for numerical labels: \renewcommand{\bibliographyitemlabel}{\@biblabel{\arabic{enumiv}}}% CONSIDER MERGING WITH PREAMBLE PART
%  to redefine the heading string ("Publications"): \renewcommand{\refname}{Articles}
\nocite{*}
\bibliographystyle{plain}
\bibliography{publications}                        
\end{document}


%% end of file `template.tex'.

